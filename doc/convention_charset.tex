% This file is part of the convention of designations for IT inventory.
%
% This convention is a free document: you can redistribute it and/or modify it
% under the terms of the GNU General Public License as published by the Free
% Software Foundation, either version 3 of the License, or (at your option) any
% later version.
%
% This document is distributed in the hope that it will be useful,but WITHOUT
% ANY WARRANTY; without even the implied warranty of MERCHANTABILITY or
% FITNESS FOR A PARTICULAR PURPOSE. See the GNU General Public License for
% more details.
%
% You should have received a copy of the GNU General Public License along with
% this document. If not, see
%
%   http://www.gnu.org/licenses/
%
%
% Copyright (C)
%   2014 Alexander Haase IT Services <support@alexhaase.de>
%

\subsection{Charset}

To avoid confusions, only the following characters are allowed:

\begin{verbatim}
0123456789ABCDEFGHJKLMNPQRSTUVWXYZ-
\end{verbatim}

\textbf{Note:} The characters ``I'' and ``O'' are \textbf{not} listed, because
they can be confused with ``1'' and ``0''. Special characters except ``-'' are
not allowed. \\


It is recommended to use only capital letters on any label that is readable by
customers. In applications any input should be handled case insensitive, so the
data input is easier for the user.
