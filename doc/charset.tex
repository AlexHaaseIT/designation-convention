% This file is part of the convention of designations for IT inventory.
%
% This convention is a free document: you can redistribute it and/or modify it
% under the terms of the GNU General Public License as published by the Free
% Software Foundation, either version 3 of the License, or (at your option) any
% later version.
%
% This document is distributed in the hope that it will be useful,but WITHOUT
% ANY WARRANTY; without even the implied warranty of MERCHANTABILITY or
% FITNESS FOR A PARTICULAR PURPOSE. See the GNU General Public License for
% more details.
%
% You should have received a copy of the GNU General Public License along with
% this document. If not, see
%
%   http://www.gnu.org/licenses/
%
%
% Copyright (C)
%   2014 Alexander Haase IT Services <support@alexhaase.de>
%

\section{Charset}

In this convention an derived alphabet of the ``Extended Hex'' Base 32 Alphabet
in RFC 4648 \cite{RFC-4648} will be used. In addition to the described alphabet,
the following characters are \textbf{not} allowed:

\begin{itemize}
	\itemsep 0pt

	\item ``I'' and ``O'' may be confused with ``1'' and ``0''
	\item ``V'' may be confused with ``u'' and ``r''\footnote{\label{foot:handwriting}especially in handwriting}
	\item ``Z'' may be confused with ``2''\footref{foot:handwriting}
\end{itemize}

Instead of forbidding ``0'', ``1'' and ``2'' as recommended in z-base-32
\cite{z-base-32}, all digits are allowed to allow integer-only designations.
For separation ``-'' is also allowed.


\begin{figure}[H]
	\label{fig:charset}

	\begin{verbatim}
		Value Encoding  Value Encoding  Value Encoding  Value Encoding  Value Encoding
		    0 0             7 7            14 E            21 M            28 U
		    1 1             8 8            15 F            22 N            29 W
		    2 2             9 9            16 G            23 P            30 X
		    3 3            10 A            17 H            24 Q            31 Y
		    4 4            11 B            18 J            25 R            32 -
		    5 5            12 C            19 K            26 S
		    6 6            13 D            20 L            27 T
	\end{verbatim}

	\caption{The allowed Alphabet and their representive integer values}
\end{figure}


Any printed label has to be printed in capital letters. In software applications
lowercase letters may be used, but it is recommended to print any information
that is accessable for customers also in \underline{big} capital letters, so
that the designation is clearly readable by them. \\

Software applications that should handle any designation case insensitive, so
the data input is easier for the user.
