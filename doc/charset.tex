% This file is part of the convention of designations for IT inventory.
%
% This convention is a free document: you can redistribute it and/or modify it
% under the terms of the GNU General Public License as published by the Free
% Software Foundation, either version 3 of the License, or (at your option) any
% later version.
%
% This document is distributed in the hope that it will be useful,but WITHOUT
% ANY WARRANTY; without even the implied warranty of MERCHANTABILITY or
% FITNESS FOR A PARTICULAR PURPOSE. See the GNU General Public License for
% more details.
%
% You should have received a copy of the GNU General Public License along with
% this document. If not, see
%
%   http://www.gnu.org/licenses/
%
%
% Copyright (C)
%   2014 Alexander Haase IT Services <support@alexhaase.de>
%

\section{Charset}

In this convention an derived alphabet of the ``Extended Hex'' Base 32 Alphabet
in RFC 4648 \cite{RFC-4648} will be used. In addition to the described alphabet,
the following characters are \textbf{\underline{not}} allowed:

\begin{itemize}
	\itemsep 0pt

	\item ``I'', ``O'': may be confused with ``1'' and ``0''\footnote{\label{foot:handwriting}especially in handwriting}
	\item ``V'': may be confused with ``u'' and ``r''\footref{foot:handwriting}
	\item ``Z'': may be confused with ``2''\footref{foot:handwriting}
\end{itemize}

Instead of forbidding ``0'', ``1'' and ``2'' as recommended in z-base-32
\cite{z-base-32}, all digits are allowed to allow integer-only designations.
For separation ``-'' is also allowed.


\begin{figure}[H]
	\label{fig:charset}

	\begin{verbatim}
		Value Encoding  Value Encoding  Value Encoding  Value Encoding  Value Encoding
		    0 0             7 7            14 E            21 M            28 U
		    1 1             8 8            15 F            22 N            29 W
		    2 2             9 9            16 G            23 P            30 X
		    3 3            10 A            17 H            24 Q            31 Y
		    4 4            11 B            18 J            25 R            32 -
		    5 5            12 C            19 K            26 S
		    6 6            13 D            20 L            27 T
	\end{verbatim}

	\caption{The allowed Alphabet and their representive integer values}
\end{figure}


Any label has to be printed in \underline{big capital} letters, so it is easier
for the customer to read them. \\

Software applications should accept lowercase letters, but transform them into
uppercase letters before processing them, so the data input is easier for the
user. If information is shown to the customer (e.g. ``information about your
Computer''), the designation should be printed in a bigger font-size.
