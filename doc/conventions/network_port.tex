% This file is part of the convention of designations for IT inventory.
%
% This convention is a free document: you can redistribute it and/or modify it
% under the terms of the GNU General Public License as published by the Free
% Software Foundation, either version 3 of the License, or (at your option) any
% later version.
%
% This document is distributed in the hope that it will be useful,but WITHOUT
% ANY WARRANTY; without even the implied warranty of MERCHANTABILITY or
% FITNESS FOR A PARTICULAR PURPOSE. See the GNU General Public License for
% more details.
%
% You should have received a copy of the GNU General Public License along with
% this document. If not, see
%
%   http://www.gnu.org/licenses/
%
%
% Copyright (C)
%   2014 Alexander Haase IT Services <support@alexhaase.de>
%

\subsection{Network Ports}
\label{chap:spec_netport}

\begin{wrapfigure}{r}{0.35\textwidth}
	Room: 212; Port: 6 \\
	Basic designation: ESC-212-6 \\

	Caclulation of the checksum:
	\begin{align*}
		\sum_{i=0}^{n} x_i &= 192 \\
		192 \mod 32 &= 7
	\end{align*}

	The complete designation will be ``ESC-212-67''.

	\caption{Example network port designation for ethernet}
\end{wrapfigure}

A designation for a network port consists of the device identifier, a room and a
port number seperated by dash ``-'' and a checksum\footref{foot:checksum}. \\

The designation should be written at the top of the network port. If the
network port has two RJ45-plugs, two letters seperated by slash will be
appended. Regular these will be ``A/B'', but if two network ports will be side
by side, they may share the port-number, but the letters will change to e.g.
``C/D''. The first letter wll always be the left plug, the second the right. For
fiber ports (or network ports that are turned $90^{\circ}$) the first will
always be top and the second the bottom. \\

The letters belong to the name of one of the two network ports (e.g. the
name of the left network port could be ``ESC-212-67A''), but they are
\underline{not} considered in the calculation of the checksum. \\


\begin{minipage}[t]{.42\textwidth}
	\begin{figure}[H]
		\begin{verbatim}

			      ########################
			      #                      #
			      #      212-67 A/B      #
			      #                      #
			      #   ######    ######   #
			      #   #    #    #    #   #
			      #   #    #    #    #   #
			      #   ######    ######   #
			      #                      #
			      #                      #
			      ########################
		\end{verbatim}

		\caption{Example network port labeling (normal)}
	\end{figure}
\end{minipage}
\hfill
\begin{minipage}[t]{.42\textwidth}
	\begin{figure}[H]
		\begin{verbatim}

			       #######################
			       #                     #
			       #      212-67 A/B     #
			       #                     #
			       #      ##########     #
			       #      ##########     #
			       #                     #
			       #      ##########     #
			       #      ##########     #
			       #                     #
			       #######################
		\end{verbatim}

		\caption{Example network port labeling (vertical)}
	\end{figure}
\end{minipage} \\

For ethernet cabeling, the device identifier will be ``ESC'', for ISDN ``ISC''.
It is irrelevant, if the singal is ethernet or ISDN, but it will be ``ISC'', if
not eight but only four wires are connected to the network-port. This implifies,
that a ISDN device may also be connected to an ESC-port, if it is switched to
an NTBA. \\

For ethernet network ports, the device identifier does not need to be printed
on an human readable label, but must be included in any digital label that is
proccessed non-human (e.g. QR code or databases).
