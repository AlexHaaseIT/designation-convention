% This file is part of the convention of designations for IT inventory.
%
% This convention is a free document: you can redistribute it and/or modify it
% under the terms of the GNU General Public License as published by the Free
% Software Foundation, either version 3 of the License, or (at your option) any
% later version.
%
% This document is distributed in the hope that it will be useful,but WITHOUT
% ANY WARRANTY; without even the implied warranty of MERCHANTABILITY or
% FITNESS FOR A PARTICULAR PURPOSE. See the GNU General Public License for
% more details.
%
% You should have received a copy of the GNU General Public License along with
% this document. If not, see
%
%   http://www.gnu.org/licenses/
%
%
% Copyright (C)
%   2014 Alexander Haase IT Services <support@alexhaase.de>
%

\subsection{Network Ports}
\label{chap:con_netport}

A designation for a network port differs from the main convention and consists
of a room and a port number seperated by dash ``-'' and a checksum.

\begin{figure}[H]
	Room: 212; Port: 6 \\
	Basic designation: 212-6 \\

	Caclulation of the checksum:
	\begin{align*}
		\sum_{i=0}^{n} result[i] &= 66 \\
		66 \mod 32 &= 2
	\end{align*}

	The complete designation will be ``212-62''.

	\caption{Example network port designation}
\end{figure}

The designation should be written at the top of the network port. \\


If your network port has two RJ45-plugs, a second row with the letters ``A'' and
``B'' under the designation is required. ``A'' will always be the left ``B'' the
right. For fiber ports (or network ports that are turned $90^{\circ}$) the top
will always be ``A'' and ``B'' always the bottom. If possible, write the letters
\underline{left} beside the RJ45- / fiber-plugs. \\


``A'' and ``B'' belong to the name of the (left or right) network port (e.g. the
name of the left network port could be ``212-62A''), but they are \underline{not}
considered in the calculation of the checksum.


\begin{minipage}[t]{.42\textwidth}
	\begin{figure}[H]
		\begin{verbatim}

			      ########################
			      #                      #
			      #        212-62        #
			      #      A        B      #
			      #   ######    ######   #
			      #   #    #    #    #   #
			      #   #    #    #    #   #
			      #   ######    ######   #
			      #                      #
			      #                      #
			      ########################
		\end{verbatim}

		\caption{Example network port labeling (normal)}
	\end{figure}
\end{minipage}
\hfill
\begin{minipage}[t]{.42\textwidth}
	\begin{figure}[H]
		\begin{verbatim}

			       #######################
			       #                     #
			       #        212-62       #
			       #                     #
			       #    A ##########     #
			       #      ##########     #
			       #                     #
			       #    B ##########     #
			       #      ##########     #
			       #                     #
			       #######################
		\end{verbatim}

		\caption{Example network port labeling (vertical)}
	\end{figure}
\end{minipage} \\
