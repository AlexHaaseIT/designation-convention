% This file is part of the convention of designations for IT inventory.
%
% This convention is a free document: you can redistribute it and/or modify it
% under the terms of the GNU General Public License as published by the Free
% Software Foundation, either version 3 of the License, or (at your option) any
% later version.
%
% This document is distributed in the hope that it will be useful,but WITHOUT
% ANY WARRANTY; without even the implied warranty of MERCHANTABILITY or
% FITNESS FOR A PARTICULAR PURPOSE. See the GNU General Public License for
% more details.
%
% You should have received a copy of the GNU General Public License along with
% this document. If not, see
%
%   http://www.gnu.org/licenses/
%
%
% Copyright (C)
%   2014 Alexander Haase IT Services <support@alexhaase.de>
%

\subsection{Device Identifier}
\label{chap:con_dev-identify}


\subsubsection{Network infrastructure}

\renewcommand{\arraystretch}{1.2}
\begin{tabular}{p{1.4cm}p{15cm}}
	APL	& ``\textbf{A}bschluss\textbf{P}unkt\textbf{L}inientechnik'' describes
		the terminating point of the wire to the telecommunication provider to
		the internal building infrastructure. \textit{\underline{This
		designation must not be set by you, but will be set by the
		telecommunication provider!} See \ref{chap:spec_provider_conn} for more
		details.} \\

	ESC & Ethernet network port. \textit{See \ref{chap:spec_netport} for more
		details.} \\

	ISC & ISDN network port. \underline{Will only be used, if only four wires
		instead of 8 are connected.} \textit{See \ref{chap:spec_netport} for
		more details.} \\

	TAE & ``\textbf{T}eilnehmer\textbf{A}nschluss\textbf{E}inheit'' descibes the
		terminating network port of the wire to the telecommunication provider.
		\textit{\underline{This designation must not be set by you, but is
		dependent on the connected APL port!} See \ref{chap:spec_provider_conn}
		for more details.} \\

	AP	& Wireles-LAN access-point \\

	RT	& Harware router \\
\end{tabular}



\subsubsection{Computer / server / virtual machines}

\begin{tabular}{p{1.4cm}p{15cm}}
	SRV	& Server \\

	WST	& Workstation, standalone computer or thinclient \\

	VC	& Virtual container \\

	VM	& Virtual machine \\
\end{tabular}



\subsubsection{Printers and similar hardware}

\begin{tabular}{p{1.4cm}p{15cm}}
	PR	& common printer (laser and inkjet) with or without multifunction (e.g.
		scanner or fax) \\

	SC	& standalone scanner \\

	FAX	& standalone fax and fax gateways \\
\end{tabular}
