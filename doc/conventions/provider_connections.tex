% This file is part of the convention of designations for IT inventory.
%
% This convention is a free document: you can redistribute it and/or modify it
% under the terms of the GNU General Public License as published by the Free
% Software Foundation, either version 3 of the License, or (at your option) any
% later version.
%
% This document is distributed in the hope that it will be useful,but WITHOUT
% ANY WARRANTY; without even the implied warranty of MERCHANTABILITY or
% FITNESS FOR A PARTICULAR PURPOSE. See the GNU General Public License for
% more details.
%
% You should have received a copy of the GNU General Public License along with
% this document. If not, see
%
%   http://www.gnu.org/licenses/
%
%
% Copyright (C)
%   2014 Alexander Haase IT Services <support@alexhaase.de>
%

\subsection{Provider connections}
\label{chap:spec_provider_conn}

The ``APL'' designations depend on designations of your telecommunication
provider. Regular your provider will tell you, at which ``APL'' port a givven
number is connected. You may need this number, if you have to connect new wires
to the APL-port, so it will be usefull to write this number down. \\

The ``TAE'' designation depends on the ``APL'' designation: If the wire is
connected to APL-port 9, then it will be ``TAE9''. \\

\textbf{Note:} \textit{For ``APL'' and ``TAE'' designations, no
checksum\footnote{\label{foot:checksum}see chapter \ref{chap:checksum}} will be
appended, otherwise there could be problems when you want to reference these by
the telecommunication provider.}
