% This file is part of the convention of designations for IT inventory.
%
% This convention is a free document: you can redistribute it and/or modify it
% under the terms of the GNU General Public License as published by the Free
% Software Foundation, either version 3 of the License, or (at your option) any
% later version.
%
% This document is distributed in the hope that it will be useful,but WITHOUT
% ANY WARRANTY; without even the implied warranty of MERCHANTABILITY or
% FITNESS FOR A PARTICULAR PURPOSE. See the GNU General Public License for
% more details.
%
% You should have received a copy of the GNU General Public License along with
% this document. If not, see
%
%   http://www.gnu.org/licenses/
%
%
% Copyright (C)
%   2014 Alexander Haase IT Services <support@alexhaase.de>
%

\section{Requirements specification}

\begin{itemize}
	\item Every designation has to be unique, otherwise two (or more) items
		could be confused. If a device is replaced, it should get a new
		designation instead of keeping the old one to distinguish the old and
		new device.

	\item A designation should not be reassigned to a new item, as long as it is
		listed in any logfile or somewhere else. Otherwise the new device could
		be confused with the old one.

	\item If this designation is used by administrators who maintain multiple
		customers, a designation should be unique per customer. In e.g.
		databases the designation should be prefixed by the customer number to
		be unique, but it shall not be printed on any label, because this is
		a redundant information and makes the designation label to long.

	\item For maintenance and customer support, the designation should be
		clearly stated and read.
		\begin{itemize}
			\item The case of the letters should be ignored, otherwise it is
				very complicated to spell.

			\item Lower- and uppercase letters should not be mixed up to not
				confuse the customer. It is recommended to use only uppercase
				letters, because they are better to read.

			\item Special characters should be avoided.
		\end{itemize}

	\item The designation should be kept short, but not too short, so that it
		could be confused with other numbers (e.g. serial number). Also it
		should not be too long, so that the customer can tell the customer
		support fast, for which device he needs help.

	\item The designation should contain a few characters to associate the
		designation with a device (e.g. a printer could have pr12345 as
		designation), so the customer and customer support can check, if the
		designation matches the right device type and it is easier for the
		customer to search for the designation.
\end{itemize}
