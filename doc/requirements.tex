% This file is part of the convention of designations for IT inventory.
%
% This convention is a free document: you can redistribute it and/or modify it
% under the terms of the GNU General Public License as published by the Free
% Software Foundation, either version 3 of the License, or (at your option) any
% later version.
%
% This document is distributed in the hope that it will be useful,but WITHOUT
% ANY WARRANTY; without even the implied warranty of MERCHANTABILITY or
% FITNESS FOR A PARTICULAR PURPOSE. See the GNU General Public License for
% more details.
%
% You should have received a copy of the GNU General Public License along with
% this document. If not, see
%
%   http://www.gnu.org/licenses/
%
%
% Copyright (C)
%   2014 Alexander Haase IT Services <support@alexhaase.de>
%

\section{Requirements}

\begin{itemize}
	\item Every designation has to be unique, otherwise two (or more) items
		could be confused.

	\item For maintenance and customer support, the designation should be
		clearly stated and read. The case of the letters should be ignored,
		otherwise it is very complicated to spell. Lower- and uppercase letters
		should not be mixed up to not confuse the customer. Special characters
		should be avoided.

	\item The designation should be kept short, but not too short, so that it
		could be confused with other numbers. Also it should not be too long, so
		that the customer can tell the customer support fast, for which device
		he needs help.

	\item The designation should contain a few characters to associate a device
		with the designation (e.g. a printer could have pr12345 as designation),
		so the customer and customer support can check, if the designation
		matches the right device type and it is easier for the customer to
		search for the designation.
\end{itemize}
