% This file is part of the convention of designations for IT inventory.
%
% This convention is a free document: you can redistribute it and/or modify it
% under the terms of the GNU General Public License as published by the Free
% Software Foundation, either version 3 of the License, or (at your option) any
% later version.
%
% This document is distributed in the hope that it will be useful,but WITHOUT
% ANY WARRANTY; without even the implied warranty of MERCHANTABILITY or
% FITNESS FOR A PARTICULAR PURPOSE. See the GNU General Public License for
% more details.
%
% You should have received a copy of the GNU General Public License along with
% this document. If not, see
%
%   http://www.gnu.org/licenses/
%
%
% Copyright (C)
%   2014 Alexander Haase IT Services <support@alexhaase.de>
%

\section{General}


\subsection{Charset}

In this convention an derived alphabet of the ``Extended Hex'' Base 32 Alphabet
in RFC 4648 \cite{RFC-4648} will be used. In addition to the described alphabet,
the following characters are \textbf{not} allowed:

\begin{itemize}
	\item ``I'' and ``O'' may be confused with ``1'' and ``0''
	\item ``V'' may be confused with ``u'' and ``r'' (especially in handwriting)
	\item ``Z'' may be confused with ``2'' (especially in handwriting)
\end{itemize}

Instead of forbidding ``0'', ``1'' and ``2'' as recommended in z-base-32
\cite{z-base-32}, all digits are allowed to allow integer-only designations.
For separation ``-'' is also allowed.


\begin{figure}[H]
	\label{charset}

	\begin{verbatim}
		Value Encoding  Value Encoding  Value Encoding  Value Encoding  Value Encoding
		    0 0             7 7            14 E            21 M            28 U
		    1 1             8 8            15 F            22 N            29 W
		    2 2             9 9            16 G            23 P            30 X
		    3 3            10 A            17 H            24 Q            31 Y
		    4 4            11 B            18 J            25 R            32 -
		    5 5            12 C            19 K            26 S
		    6 6            13 D            20 L            27 T
	\end{verbatim}

	\caption{The allowed Alphabet and their representive integer values}
\end{figure}


It is recommended to use only capital letters on any label that is readable by
customers. In applications any input should be handled case insensitive, so the
data input is easier for the user.



\subsection{Checksum}

To avoid errors when transferring the designation, a check digit is appended to
each designation. \\

The checksum is computed equivalent to the check digit of a
GTIN\footnote{Global Trade Item Number} \cite{ean_checksum}. It results from
the modulo 32 (allowed charset without ``-'') of the summed from behind
characters, which are alternately multiplied by weight of 3 and 1. \\

For the computation of the checksum, each character is interpreted as its
representive integer value (see \ref{charset}). The resulting checksum will be
encoded back to its representive character as defined in \ref{charset}. \\


\begin{multicols}{2}
	\begin{figure}[H]
		\centering

		\begin{tabular}{r|c|c|c|c|c|c}
			i          & 1  & 2  & 3 & 4 & 5 & 6 \\
			\hline
			$x_i$      & A  & B  & 1 & 2 & 3 & 4 \\
			weight     & 1  & 3  & 1 & 3 & 1 & 3 \\
			\hline
			$result_i$ & 10 & 33 & 1 & 6 & 3 & 12 \\
		\end{tabular}

		\begin{align*}
			\sum_{i=0}^{n} result[i] &= 65 \\
			65 \mod 32 &= 1
		\end{align*}

		Complete designation: ``AB12341''.

		\caption{Example for even $n$}
	\end{figure}

	\begin{figure}[H]
		\centering

		\begin{tabular}{r|c|c|c|c|c|c|c}
			i          & 1  & 2  & 3 & 4 & 5 & 6 & 7 \\
			\hline
			$x_i$      & A  & B  & 1 & 2 & 3 & 4 & 5 \\
			weight     & 3  & 1  & 3 & 1 & 3 & 1 & 3 \\
			\hline
			$result_i$ & 30 & 11 & 3 & 2 & 9 & 4 & 15 \\
		\end{tabular}

		\begin{align*}
			\sum_{i=0}^{n} result[i] &= 74 \\
			74 \mod 32 &= 10 \hat{=} A
		\end{align*}

		Complete designation: ``AB12345A''.

		\caption{Example for odd $n$}
	\end{figure}
\end{multicols}


This algorithm could be written as mathematical function for the string $x$ with
length $n$:
\begin{align*}
	f(x, n) &= \left(\sum_{i=1}^{n} x_i \cdot \left(1 + 2 \cdot \left( \left(
		n - i + 1 \right) \mod 2 \right) \right) \right) \mod 32
\end{align*}
