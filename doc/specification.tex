% This file is part of the convention of designations for IT inventory.
%
% This convention is a free document: you can redistribute it and/or modify it
% under the terms of the GNU General Public License as published by the Free
% Software Foundation, either version 3 of the License, or (at your option) any
% later version.
%
% This document is distributed in the hope that it will be useful,but WITHOUT
% ANY WARRANTY; without even the implied warranty of MERCHANTABILITY or
% FITNESS FOR A PARTICULAR PURPOSE. See the GNU General Public License for
% more details.
%
% You should have received a copy of the GNU General Public License along with
% this document. If not, see
%
%   http://www.gnu.org/licenses/
%
%
% Copyright (C)
%   2014 Alexander Haase IT Services <support@alexhaase.de>
%

\section{Specification}

\textit{All designations are constructed according to the following pattern, if
not otherwise specified in the relevant specification.} \\


A designation starts with an one to three alphanumerical ``device identifier''
followed by a alphanumeric ``ID string'' and terminated by the checksum as
defined in chapter \ref{chap:checksum}. The ``device identifier''
\underline{must} begin with letters and may be continued with a number. If
numbers are used in the ``device identifier'', a hypen must seperate the
``device identifier'' and ``ID string''. \\

The maximum length of the whole designation \underline{without hyphens} is 10. A
group of characters between hyphens may not be longer than 4 characters.



%
% include specific conventions
%

% This file is part of the convention of designations for IT inventory.
%
% This convention is a free document: you can redistribute it and/or modify it
% under the terms of the GNU General Public License as published by the Free
% Software Foundation, either version 3 of the License, or (at your option) any
% later version.
%
% This document is distributed in the hope that it will be useful,but WITHOUT
% ANY WARRANTY; without even the implied warranty of MERCHANTABILITY or
% FITNESS FOR A PARTICULAR PURPOSE. See the GNU General Public License for
% more details.
%
% You should have received a copy of the GNU General Public License along with
% this document. If not, see
%
%   http://www.gnu.org/licenses/
%
%
% Copyright (C)
%   2014 Alexander Haase IT Services <support@alexhaase.de>
%

\subsection{Device Identifier}
\label{chap:con_dev-identify}


\subsubsection{Network infrastructure}

\renewcommand{\arraystretch}{1.2}
\begin{savenotes}
\begin{tabular}{p{1.4cm}p{15cm}}
	APL\footnote{\textbf{A}bschluss\textbf{P}unkt\textbf{L}inientechnik} & Describes
		the terminating point of the wire to the telecommunication provider to
		the internal building infrastructure. \textit{\underline{This
		designation must not be set by you, but will be set by the
		telecommunication provider!} See \ref{chap:spec_provider_conn} for more
		details.} \\

	SC2 & PSTN network port. \underline{Will be used, if only two wires instead
		of eight are connected.} \textit{See \ref{chap:spec_netport} for more
		details.} \\

	SC4 & ISDN network port. \underline{Will be used, if only four wires instead
		of eight are connected.} \textit{See \ref{chap:spec_netport} for more
		details.} \\

	SC8 & Ethernet network port. \textit{See \ref{chap:spec_netport} for more
		details.} \\

	TAE\footnote{\textbf{T}eilnehmer\textbf{A}nschluss\textbf{E}inheit} & Descibes
		the terminating network port of the wire to the telecommunication
		provider. \textit{\underline{This designation must not be set by you,
		but is dependent on the connected APL port!} See
		\ref{chap:spec_provider_conn} for more details.} \\

	AP	& Wireles-LAN access-point \\

	RT	& Harware router \\
\end{tabular}
\end{savenotes}


\subsubsection{Computer / server / virtual machines}

\begin{savenotes}
\begin{tabular}{p{1.4cm}p{15cm}}
	C	& Virtual container (like LXC\footnote{\textbf{L}inu\textbf{X} \textbf{C}ontainers}) \\

	D	& Virtual machine (like domains in XEN) \\

	S	& Server \\

	W	& Workstation, standalone computer or thinclient \\
\end{tabular}
\end{savenotes}



\subsubsection{Printers and similar hardware}

\begin{tabular}{p{1.4cm}p{15cm}}
	PR	& common printer (laser and inkjet) with or without multifunction (e.g.
		scanner or fax) \\

	FAX	& standalone fax and fax gateways \\
\end{tabular}

% This file is part of the convention of designations for IT inventory.
%
% This convention is a free document: you can redistribute it and/or modify it
% under the terms of the GNU General Public License as published by the Free
% Software Foundation, either version 3 of the License, or (at your option) any
% later version.
%
% This document is distributed in the hope that it will be useful,but WITHOUT
% ANY WARRANTY; without even the implied warranty of MERCHANTABILITY or
% FITNESS FOR A PARTICULAR PURPOSE. See the GNU General Public License for
% more details.
%
% You should have received a copy of the GNU General Public License along with
% this document. If not, see
%
%   http://www.gnu.org/licenses/
%
%
% Copyright (C)
%   2014 Alexander Haase IT Services <support@alexhaase.de>
%

\subsection{Provider connections}
\label{chap:spec_provider_conn}

The ``APL'' designations depend on designations of your telecommunication
provider. Regular your provider will tell you, at which ``APL'' port a givven
number is connected. You may need this number, if you have to connect new wires
to the APL-port, so it will be usefull to write this number down. \\

The ``TAE'' designation depends on the ``APL'' designation: If the wire is
connected to APL-port 9, then it will be ``TAE9''. \\

\textbf{Note:} \textit{For ``APL'' and ``TAE'' designations, no
checksum\footnote{\label{foot:checksum}see chapter \ref{chap:checksum}} will be
appended, otherwise there could be problems when you want to reference these by
the telecommunication provider.}

% This file is part of the convention of designations for IT inventory.
%
% This convention is a free document: you can redistribute it and/or modify it
% under the terms of the GNU General Public License as published by the Free
% Software Foundation, either version 3 of the License, or (at your option) any
% later version.
%
% This document is distributed in the hope that it will be useful,but WITHOUT
% ANY WARRANTY; without even the implied warranty of MERCHANTABILITY or
% FITNESS FOR A PARTICULAR PURPOSE. See the GNU General Public License for
% more details.
%
% You should have received a copy of the GNU General Public License along with
% this document. If not, see
%
%   http://www.gnu.org/licenses/
%
%
% Copyright (C)
%   2014 Alexander Haase IT Services <support@alexhaase.de>
%

\subsection{Network Ports}
\label{chap:spec_netport}

\begin{wrapfigure}{r}{0.35\textwidth}
	Room: 212; Port: 6 \\
	Basic designation: SC8-212-6 \\

	Caclulation of the checksum:
	\begin{align*}
		\sum_{i=0}^{n} x_i &= 217 \\
		217 \mod 32 &= 25 \hat{=} R
	\end{align*}

	The complete designation will be ``SC8-212-6R''.

	\caption{Example network port designation for ethernet}
\end{wrapfigure}

A designation for a network port consists of the device identifier, a room and a
port number seperated by hyphen ``-'' and a checksum\footref{foot:checksum}. \\

The designation should be written at the top of the network port. If the
network port has two RJ45-plugs, two letters seperated by slash will be
appended. Regular these will be ``A/B'', but if two network ports will be side
by side, they may share the port-number, but the letters will change to e.g.
``C/D''. The first letter wll always be the left plug, the second the right. For
fiber ports (or network ports that are turned $90^{\circ}$) the first will
always be top and the second the bottom. \\

The letters belong to the name of one of the two network ports (e.g. the
name of the left network port could be ``SC8-212-6RA''), but they are
\underline{not} considered in the calculation of the checksum. \\

For ethernet cabeling, the device identifier will be ``SC8'', for ISDN ``SC4''
and for old PSTN (two wires) ``SC2''. It is not relevant, if the singal is
ethernet or ISDN, but it will be ``SC4'', if not eight but only four wires are
connected to the network-port. This implies, that a ISDN device may be connected
to an ESC-port, but of course this port has to be switched to an $S_0$ bus. \\

At network ports with the device identifier ``SC8'' the device identifier does
not need to be printed on an human readable label, but must be included in any
digital label that is proccessed non-human (e.g. QR code or databases). e.g.
instead of printing ``SC8-212-6RA'' you may print ``212-6RA''.

\begin{minipage}[t]{.42\textwidth}
	\begin{figure}[H]
		\begin{verbatim}

			      ########################
			      #                      #
			      #      212-6R A/B      #
			      #                      #
			      #   ######    ######   #
			      #   #    #    #    #   #
			      #   #    #    #    #   #
			      #   ######    ######   #
			      #                      #
			      #                      #
			      ########################
		\end{verbatim}

		\caption{Example network port labeling (normal)}
	\end{figure}
\end{minipage}
\hfill
\begin{minipage}[t]{.42\textwidth}
	\begin{figure}[H]
		\begin{verbatim}

			       #######################
			       #                     #
			       #      212-6R A/B     #
			       #                     #
			       #      ##########     #
			       #      ##########     #
			       #                     #
			       #      ##########     #
			       #      ##########     #
			       #                     #
			       #######################
		\end{verbatim}

		\caption{Example network port labeling (vertical)}
	\end{figure}
\end{minipage}

